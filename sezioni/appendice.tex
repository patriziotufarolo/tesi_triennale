\appendix
\chapter{Driver di sonda - Isolamento della rete e firewalling}
\begin{python}
__author__ = 'Patrizio Tufarolo'
__email__ = 'patrizio.tufarolo@studenti.unimi.it'


import time,uuid,subprocess

from testagent.probe import Probe

from neutronclient.v2_0 import client as NeutronClient
from novaclient.client import Client as NovaClient
from keystoneclient.auth.identity import v2 as KeystoneClient
from keystoneclient import session as KeystoneSession
from libnmap.process import NmapProcess
from libnmap.parser import NmapParser, NmapParserException

class NetworkSecurityGroup(Probe):
	def nullRollback(self,inputData):
		return inputData

	def createNeutronClient(self,inputData):
		self.neucl = NeutronClient.Client(auth_url=self.testinstances["OpenStackConfig"]["OS_AUTH_URL"],
						username=self.testinstances["OpenStackConfig"]["OS_USERNAME"],
						password=self.testinstances["OpenStackConfig"]["OS_PASSWORD"],
						tenant_name=self.testinstances["OpenStackConfig"]["OS_TENANT_NAME"],
						region_name=self.testinstances["OpenStackConfig"]["OS_REGION_NAME"])
		self.keystonecl = KeystoneClient.Password(auth_url=self.testinstances["OpenStackConfig"]["OS_AUTH_URL"],
						username=self.testinstances["OpenStackConfig"]["OS_USERNAME"],
						password=self.testinstances["OpenStackConfig"]["OS_PASSWORD"],
						tenant_name=self.testinstances["OpenStackConfig"]["OS_TENANT_NAME"])
		sess = KeystoneSession.Session(auth=self.keystonecl)
		self.novacl = NovaClient(3,session=sess)
		return inputData


	def deployHoneypot(self,inputData):
		name = "honeypot_" + str(uuid.uuid4())
		print("Deploying honeypot "+ str(name))
		imagename = self.testinstances["DeployHoneypot"]["Image"]
		print(imagename)
		flavorname = self.testinstances["DeployHoneypot"]["Flavor"]
		networkname = self.testinstances["DeployHoneypot"]["Network"]
		checktcp = self.testinstances["Configuration"]["CheckTCP"]
		checkudp = self.testinstances["Configuration"]["CheckUDP"]
		sgname = self.testinstances["DeployHoneypot"]["SecurityGroup"]

		openAllPorts = """#!/bin/bash\n"""
		if checktcp == "True":
			tcpminport = int(self.testinstances["Configuration"]["TCPMinPort"])
			tcpmaxport = int(self.testinstances["Configuration"]["TCPMaxPort"])
			openAllPorts += '''
			echo "Adding iptables TCP redirect\n"
			iptables -t nat -A PREROUTING -p tcp --dport '''+str(tcpminport)+':'+str(tcpmaxport-1)+' -j REDIRECT --to-ports '+str(tcpmaxport)+'''\n
			echo "Running TCP server in background on port '''+str(tcpmaxport)+'''\n"
			echo -e "import socket, sys; sock=socket.socket(socket.AF_INET, socket.SOCK_STREAM); sock.bind(('','''+str(tcpmaxport)+''')); sock.listen(1); \nwhile True: s,a = sock.accept()\n" | python &
			\n'''

		if checkudp == "True":
			udpminport = int(self.testinstances["Configuration"]["UDPMinPort"])
			udpmaxport = int(self.testinstances["Configuration"]["UDPMaxPort"])
			openAllPorts += '''
			echo "Adding iptables UDP redirect"\n
			iptables -t nat -A PREROUTING -p udp --dport '''+str(udpminport)+':'+str(udpmaxport-1)+' -j REDIRECT --to-ports '+str(udpmaxport)+'''\n
			echo "Running python server in background on port''' + str(udpmaxport) + ''' udp\n"
			echo -e "import socket, sys; sock=socket.socket(socket.AF_INET, socket.SOCK_DGRAM); sock.bind((,'''+str(udpmaxport)+''')); \nwhile True: d,a = sock.recvfrom(4096)\n" | python &'''
		print openAllPorts
		image = self.novacl.images.find(name=imagename)
		flavor = self.novacl.flavors.find(name=flavorname)
		net = self.novacl.networks.find(label=networkname)
		server = self.novacl.servers.create(name,image,flavor,security_groups=[sgname],nics=[{'net-id': net.id}],userdata=openAllPorts)
		print("Deployed honeypot")
		print(server)
		self.server_id = server.id
		return inputData

	def removeHoneypot(self,inputData):
		print(self.novacl.servers.delete(self.server_id))
		print("Removed honeypot")
		return inputData

	def pollingHoneypot(self,inputData):
		try:
			timeout = time.time() + int(self.testinstances["DeployHoneypot"]["Timeout"]) or 120
		except:
			timeout = time.time() + 120
		status = self.novacl.servers.get(self.server_id).status
		while status == "BUILD":
			time.sleep(5)
			if timeout < time.time():
				raise Exception("Timeout!")
			status = self.novacl.servers.get(self.server_id).status
		if status != "ACTIVE":
			raise Exception("Failed to boot honeypot. Status is " + status + ". Please look at Openstack logs for more info.")
		print("Honeypot ready")
		return inputData


	def createFloatingIp(self,inputData):
		self.floating_ip = self.novacl.floating_ips.create(pool=self.testinstances['DeployHoneypot']['FloatingIPPool'])
		print("Created floating ip")
		return inputData

	def deleteFloatingIp(self,inputData):
		self.novacl.floating_ips.delete(self.floating_ip)
		print("Deleted floating ip")
		return inputData


	def attachFloatingIp(self,inputData):
		time.sleep(5)
		self.novacl.servers.get(self.server_id).add_floating_ip(self.floating_ip)
		return inputData
	def deattachFloatingIp(self,inputData):
		self.novacl.servers.get(self.server_id).remove_floating_ip(self.floating_ip)
		return inputData

	def parseSecurityGroup(self,inputData):
		allSecurityGroups = self.novacl.servers.get(self.server_id).list_security_group()
		allSecurityGroups_details = []
		for sg in allSecurityGroups:
			allSecurityGroups_details.append(self.neucl.show_security_group(str(sg)))
		print(allSecurityGroups_details)
		dict_tcp = {}
		dict_udp = {}
		list_tcp = []
		list_udp = []
		icmp = 0
		for sg in allSecurityGroups_details:
			for rule in sg['security_group']['security_group_rules']:
				if rule["direction"] == "ingress":
					if rule["protocol"] == "icmp":
						icmp = 1 if icmp == 0 else icmp
					
					elif rule["port_range_min"] and rule["port_range_max"]:
						for i in range(rule["port_range_min"],rule["port_range_max"]+1):
							if self.testinstances["Configuration"]["CheckTCP"] and rule["protocol"] == "tcp" and i >= int(self.testinstances["Configuration"]["TCPMinPort"]) and i <= int(self.testinstances["Configuration"]["TCPMaxPort"]):
								dict_tcp[i] = 1
							if self.testinstances["Configuration"]["CheckUDP"] and rule["protocol"] == "udp" and i >= int(self.testinstances["Configuration"]["UDPMinPort"]) and i <= int(self.testinstances["Configuration"]["UDPMaxPort"]):	
								dict_udp[i] = 1
		if self.testinstances["Configuration"]["CheckTCP"]:
			print("Parsed TCP:")
			print(dict_tcp)
		if self.testinstances["Configuration"]["CheckUDP"]:
			print("Parsed UDP:")
			print dict_udp

		return icmp, dict_tcp, dict_udp

	def doRealScan(self,inputData):
		time.sleep(120)
		icmp_flag, dict_tcp, dict_udp = inputData
		def do_scan(target,options):
			command = ["/usr/bin/nmap", "-oX", "-"] + options + [str(target)]
			print("Executing nmap Command")
			print(command)
			output = subprocess.check_output(command)
			print output
			return NmapParser.parse(output)
		ip = str(self.floating_ip.ip)
		options = ""
		new_options = []
		if self.testinstances["NMapScan"]["Pn"] == "True":
			options += " -Pn"
			new_options.append("-Pn")

		if self.testinstances["Configuration"]["CheckTCP"] == "True":
			tcp_options = options + " -p"+self.testinstances["Configuration"]["TCPMinPort"]+"-"+self.testinstances["Configuration"]["TCPMaxPort"]+" -s"
			tcp_options += "S" if self.testinstances["NMapScan"]["TCPSynScan"] == "True" else "T"
			
			new_tcp_options = list(new_options)
			new_tcp_options.append("-p"+self.testinstances["Configuration"]["TCPMinPort"]+"-"+self.testinstances["Configuration"]["TCPMaxPort"])
			if self.testinstances["NMapScan"]["TCPSynScan"] == "True":
				new_tcp_options.append("-sS")
			else:
				new_tcp_options.append("-sT")

			tcpScanResults = do_scan(ip,new_tcp_options)
			print("tcpScanResults")
			print(tcpScanResults)
			for host in tcpScanResults.hosts:
				for port, protocol in host.get_open_ports():
					if protocol == "tcp" and port not in dict_tcp:
						return False


		if self.testinstances["Configuration"]["CheckUDP"] == "True":
			udp_options = options + " -p"+self.testinstances["Configuration"]["UDPMinPort"]+"-"+self.testinstances["Configuration"]["UDPMaxPort"]+" -sU"	
			udpScanResults = do_scan(ip,udp_options)
			print("udpScanResults")
			print(udpScanResults)
			for host in udpScanResults.hosts:
				for port, protocol in host.get_open_ports():
					if protocol == "udp" and port not in dict_udp:
						return False

		return True


	def checkResult(self,inputData):
		self.deattachFloatingIp(inputData)
		self.deleteFloatingIp(inputData)
		self.removeHoneypot(inputData)
		return inputData

	def appendAtomics(self):
		self.appendAtomic(self.createNeutronClient,self.nullRollback)
		self.appendAtomic(self.deployHoneypot,self.removeHoneypot)
		self.appendAtomic(self.createFloatingIp,self.deleteFloatingIp)
		self.appendAtomic(self.attachFloatingIp,self.deattachFloatingIp)
		self.appendAtomic(self.pollingHoneypot,self.nullRollback)
		self.appendAtomic(self.parseSecurityGroup,self.nullRollback)
		self.appendAtomic(self.doRealScan,self.nullRollback)
		self.appendAtomic(self.checkResult,self.nullRollback)

probe = NetworkSecurityGroup()
\end{python}
\label{app:isolamento}
\chapter{Driver di sonda - Confidenzialità dei file}
\begin{python}
__author__ = 'Patrizio Tufarolo'
__email__ = 'patrizio.tufarolo@studenti.unimi.it'

from testagent.probe import Probe
import paramiko
import os
import socket

class CFile(Probe):

	def parseChmod(self,chmod):
		chmod = str(chmod)
		a = [['0', '0', '0'], ['0', '0', '0'], ['0', '0', '0']]
		for i in range(0, 3):
			digit = int(chmod[i])
			for j in range(0, 3):
				a[i][j] = str(int(digit / pow(2, 2 - j)) % 2)
		return a

	class SSHClient(object):

		def __init__(self, hostname, username, keyfile, keypassphrase, server_key_file, server_key_policy, port=22):
			self.hostname = hostname
			self.ip = socket.gethostbyname(hostname)
			self.username = username
			self.keyfile = keyfile
			self.key = paramiko.RSAKey.from_private_key_file(keyfile,password=keypassphrase)
			self.port = port
			self.connected = False
			self.client = paramiko.SSHClient()
			self.client.load_host_keys(server_key_file)
			if server_key_policy == "autoadd":
				self.client.set_missing_host_key_policy(paramiko.AutoAddPolicy())
			elif server_key_policy == "reject":
				self.client.set_missing_host_key_policy(paramiko.RejectPolicy())

		def connect(self):
			try:
				print("Connecting to the target through SSH")
				self.client.connect(self.hostname, self.port, self.username, pkey=self.key)
			except:
				pass

			self.connected = True
			return self

		def executeCommand(self, command=""):
			if (self.connected == False):
				raise Exception, "Not connected"
			if (command == ""):
				raise Exception, "No command"
			stdin, stdout, stderr = self.client.exec_command(command)
			out = stdout.readlines()
			output = ""
			for line in out:
				if line != '':
					output = output + line
			return output

		def close(self):
			self.client.close()

	def rollback():
		try:
			self.ssh.close()
		except:
			pass

	def configuration(self, inputs):
		print("Started configuration phase")
		host = self.testinstances["configuration"]["host"]
		try:
			port = int(self.testinstances["configuration"]["port"]) or 22
		except:
			port = 22
		ssh_key_path = self.testinstances["configuration"]["ssh_key_path"]
		ssh_key_passphrase = self.testinstances["configuration"]["ssh_key_passphrase"]
		server_key_file = self.testinstances["configuration"]["server_key_file"]
		try:
			server_key_policy = self.testinstances["configuration"]["server_key_policy"] or "autoadd"
		except:
			server_key_policy = "autoadd"

		file_path = self.testinstances["configuration"]["file_path"]
		file_chmod = self.testinstances["configuration"]["file_chmod"]
		print("Completed configuration phase")
		return host, port, ssh_key_path, ssh_key_passphrase, server_key_file, server_key_policy, file_path, file_chmod


	def chmod_test(self,inputs):
		print("Starting CHMOD test")
		username = self.testinstances["test_owner"]["username"]
		print("Username set")
		hostname, port, ssh_key_path, ssh_key_passphrase, server_key_file, server_key_policy, file_path, chmod = inputs
		print("Now let's SSH into the target")
		self.ssh = self.SSHClient(hostname,username,ssh_key_path,ssh_key_passphrase, server_key_file, server_key_policy, port)
		self.ssh.connect()
		print("Connected")
		filelist = self.ssh.executeCommand("stat -c \"%a\" " + file_path)
		for eachFile in filelist.split("\n"):
			if eachFile and eachFile != chmod:
				print("Test failed")
				try:
					print("Closing SSH connection")
					self.ssh.close()
				except:
					pass
				return False, inputs
		try:
			print("Closing SSH Connection")
			self.ssh.close()
		except:
			pass
		print("Test succeeded")
		return True, inputs


	def chmod_rw_test(self, username, hostname, port, ssh_key_path, ssh_key_passphrase, server_key_file, server_key_policy, file_path, hasToBeR, hasToBeW):
		print("Connecting through SSH")
		self.ssh = self.SSHClient(hostname, username, ssh_key_path, ssh_key_passphrase, server_key_file, server_key_policy, port)
		self.ssh.connect()
		print("Getting file list")
		filelist = self.ssh.executeCommand("stat  -c \"%n\" " + file_path)
		filelistsplitted = filelist.split("\n")

		for eachFile in filelistsplitted[0:len(filelistsplitted)-1]:
			print(self.inject_read(eachFile))
			output = self.ssh.executeCommand(self.inject_read(eachFile))
			output = output.strip("\n")
			if (hasToBeR == '1' and output == 'False') or (hasToBeR == '0' and output == 'True'):
				print("Test failed")
				try:
					print("Closing SSH Connection")
					self.ssh.close()
				except:
					pass
				return False
			output = self.ssh.executeCommand(self.inject_write(eachFile))

			output = output.strip("\n")
			print ("Inj write "+ output)
			if (hasToBeW == '1' and output == 'False') or (hasToBeW == '0' and output == 'True'):
				print("Test failed")
				try:
					print("Closing SSH Connection")
					self.ssh.close()
				except:
					pass
				return False
		try:
			print("Closing SSH Connection")
			self.ssh.close()
		except:
			pass
		print("Test succeded")
		return True


	def run_test_owner(self,inputs):
		print("Starting OWNER R/W test")
		if not inputs:
			return False
		previousResult, inputs = inputs
		if not previousResult:
			return False
		hostname, port, ssh_key_path, ssh_key_passphrase, server_key_file, server_key_policy, file_path, chmod = inputs
		username = self.testinstances["test_owner"]["username"]
		chmod = self.parseChmod(chmod)[0]
		hasToBeR = chmod[0]
		hasToBeW = chmod[1]
		return self.chmod_rw_test(username, hostname, port, ssh_key_path, ssh_key_passphrase, server_key_file, server_key_policy, file_path, hasToBeR, hasToBeW), inputs

	def run_test_group(self,inputs):
		print("Starting GROUP R/W test")
		if not inputs:
			return False
		previousResult, inputs = inputs
		if not previousResult:
			return False

		hostname, port, ssh_key_path, ssh_key_passphrase, server_key_file, server_key_policy, file_path, chmod = inputs
		username = self.testinstances["test_group"]["username"]
		chmod = self.parseChmod(chmod)[1]
		hasToBeR = chmod[0]
		hasToBeW = chmod[1]
		return self.chmod_rw_test(username, hostname, port, ssh_key_path, ssh_key_passphrase, server_key_file, server_key_policy, file_path, hasToBeR, hasToBeW), inputs

	def run_test_other(self, inputs):
		print("Starting OTHER R/W test")
		if not inputs:
			return False
		previousResult, inputs = inputs
		if not previousResult:
			return False
			
		hostname, port, ssh_key_path, ssh_key_passphrase, server_key_file, server_key_policy, file_path, chmod = inputs
		username = self.testinstances["test_other"]["username"]
		chmod = self.parseChmod(chmod)[2]
		hasToBeR = chmod[0]
		hasToBeW = chmod[1]
		return self.chmod_rw_test(username, hostname, port, ssh_key_path, ssh_key_passphrase, server_key_file, server_key_policy, file_path, hasToBeR, hasToBeW)


	def inject_read(self, file_path):
		print("Injecting CHECK READ script into host")
		return 'echo -e "try:\\n\\topen(\\"' + file_path + '\\",\\"r\\");\\n\\tprint(\\"True\\")\\nexcept IOError:\\n\\tprint(\\"False\\")" | python'

	def inject_write(self, file_path):
		print("Injecting CHECK WRITE script into host")
		return 'echo -e "try:\\n\\topen(\\"' + file_path + '\\",\\"a\\");\\n\\tprint(\\"True\\")\\nexcept IOError:\\n\\tprint(\\"False\\")" | python'


	def appendAtomics(self):
		self.appendAtomic(self.configuration, self.rollback)
		self.appendAtomic(self.chmod_test, self.rollback)
		self.appendAtomic(self.run_test_owner, self.rollback)
		self.appendAtomic(self.run_test_group, self.rollback)
		self.appendAtomic(self.run_test_other, self.rollback)


probe = CFile()
\end{python}
\label{app:confid}
